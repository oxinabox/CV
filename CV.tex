\documentclass[11pt,a4paper,sans]{moderncv}

% moderncv themes
\moderncvstyle{classic}
\moderncvcolor{green}
\renewcommand{\familydefault}{\sfdefault} 
\nopagenumbers{}
\hyphenpenalty=1000
\sethintscolumntowidth{2016 April -- 2019 April}

% adjust the page margins
\usepackage[scale=.75,top=2cm, bottom=1cm]{geometry}


% personal data
\name{Frames}{White}
\photo[64pt][0.4pt]{portrait}
\title{Curriculum Vitae}
\email{oxinabox@ucc.asn.au}
\homepage{oxinabox.net}
\social[linkedin]{frames-catherine-white-46b9a035/}
\social[github]{oxinabox}
\extrainfo{%
	\httplink[\faStackOverflow{} frames-white]{stackoverflow.com/users/179081/}%
}

% -------------------------------------------------
% bibliography adjustements
%   to show numerical labels in the bibliography (default is to show no labels)
\makeatletter\renewcommand*{\bibliographyitemlabel}{\@biblabel{\arabic{enumiv}}}\makeatother
%   to redefine the bibliography heading string ("Publications")
%\renewcommand{\refname}{Articles}

% https://tex.stackexchange.com/a/70946 + https://github.com/moderncv/moderncv/blob/d8bc48733c400ccc93f1b03b3eafe08d5b3220e6/moderncvbodyi.sty#L132
\makeatletter
\long\def\bibNote#1{\gdef\@bibNote{\item[]{\normalsize#1}}}
\renewenvironment{thebibliography}[1]%
  {%
    \bibliographyhead{\refname}%
    \begin{list}{\bibliographyitemlabel}%
      {%
        \setlength{\topsep}{0pt}%
        \setlength{\labelwidth}{\hintscolumnwidth}%
        \setlength{\labelsep}{\separatorcolumnwidth}%
        \leftmargin\labelwidth%
        \advance\leftmargin\labelsep%
        \@openbib@code%
        \usecounter{enumiv}%
        \let\p@enumiv\@empty%
        \renewcommand\theenumiv{\@arabic\c@enumiv}}%
        \sloppy%
        \clubpenalty4000%\@clubpenalty \clubpenalty%
        \widowpenalty4000%
        \sfcode`\.\@m%
        \sfcode `\=1000\relax
		\@bibNote\settowidth\labelwidth{\@biblabel{#1}}	
	}%
  {%
    \def\@noitemerr{\@latex@warning{Empty `thebibliography' environment}}%
    \end{list}}
\makeatother

%----------------------------------------------------------------------------------
%            content
%----------------------------------------------------------------------------------
\begin{document}
\makecvtitle

\section{EDA and Compiler Engineer at JuliaHub (2023~Feb~--~2024~July)}
I worked on the CedarEDA electronic simulation tool, where I focused on compiler transformations for solving differential algebraic equations.
A particularly important subset of this work was automatic differentiation, facilitating end-to-end differentiation of the simulation allowing, for example, differentiation of the rise-time with regards to the transistor width.
More broadly, I delved deep into the Julia compiler internals which were heavily reused in this tool, including inference, constant propagation and abstract interpretation.


\section {Research Software Engineering at Invenia Labs(2018~Oct~--~2023~Feb)}
I was a founding member of research software engineering team, which existed to handle tasks to technical for researchers and too scientific for developers.
where I developed extensive experience in API design, machine learning, constrained and unconstrained optimisation, and automatic differentiation.
As team grew, I took on a management role (both people management and project management), while still maintaining significant technical contributions and up-skilling many junior developers.
From July 2022 on wards I was also the acting program manager, and eventually took on several other tasks from senior management.


\section{Full History}
%\cventry{years}{Job}{Employer}{localization}{grade}{description}
\cventry{2023 Feb -- 2024 July}{EDA and Compiler Engineer}{JuliaHub}{Remote}{}{}
\cventry{2021 July -- 2023 Feb }{Research Software Engineering Group Lead}{Invenia Labs}{Cambridge, UK}{}{}
\cventry{2018 Oct -- July 2021}{Research Software Engineer}{Invenia Labs}{Cambridge, UK}{}{}
\cventry{2015 March-- 2018 Oct}{PhD Candidature}{The University of Western Australia}{}{}{Thesis: ``On the surprising capacity of linear combinations of embeddings for natural language processing''. (Machine learning for natural language processing)}
\cventry{2014 Nov -- 2018 Oct}{Research and Teaching Assistant}{The University of Western Australia}{Perth, WA}{(Casual)}{}
\cventry{2011 July -- 2012 March}{Agile Platforms Developer}{Bankwest}{Perth, WA}{Australian Computing Society Workplace Integrated Learning Scholarship}{}
\cventry{2009--2014}{Bachelor of Engineering, with Honours}{The University of Western Australia}{}{}{(Majoring in Electrical and Electronic Engineering)}
\cventry{2009--2014}{Bachelor of Computer and Mathematical Sciences}{The University of Western Australia}{}{}{(Double major in Pure Mathematics, and Computation)}

\newpage

\section{Awards}
\cvitem{2022 Julia Community Prize}{``For her many technical and community contributions across the Julia ecosystem.''}
\cvitem{2021 Best Poster Award}{AbstractDifferentiation.jl: Backend-Agnostic Differentiable Programming in Julia, Frank Schäfer et al. (NeurIPS Differentiable Programming Workshop)}
\cvitem{2020 DSTG Best Contribution to Science Award}{WEmbSim: A Simple yet Effective Metric for Image Captioning, Naeha Sharif et al. (International Conference on Digital Image Computing: Techniques and Applications)}
\cvitem{2016 Best Student Paper Award}{Generating Bags of Words from the Sums of their Word Embeddings, White et al (Conference on Intelligent Text Processing and Computational Linguistics)}


\section{Organisations}
\cventry{2016 April -- 2019 April}{Administrator of the Board}{Western Australian Science Fiction Foundation}{}{}{}
\cventry{2015 March -- \raisebox{0.9pt}{$\infty$}}{Honarary Life Member}{Unigames}{(UWA Student Society)}{}{}

\section{Open Source}
I am the a major contributor to to vastly too many projects to list here, both professionally and in my own time.
A more complete list is available at \url{https://github.com/oxinabox}.
Some of the more notable include: being the comaintainer of \emph{TensorFlow.jl} (now deprecated),
being the maintainer of \emph{DataStructures.jl}, being the creator and maintainer of \emph{LoggingExtras.jl} 
and being the leader of the \emph{ChainRules.jl} project.
The last of which is a moderately sized collaboration with folk from several different institutions to produce fundamental automatic differentiation tooling.
It has well-over 100 directly dependent packages and over 2500 indirectly dependent packages.
You can find various talks on this, and my other open source projects on YouTube, by searching for ``Frames White JuliaCon'' and ``Frames White EuroAD''.



%%%%%%%%%%%%%%%%%%%%%%%%%%%%%%
\clearpage
\sethintscolumntowidth{[1234]}
\small
\nocite{*}
\bibliographystyle{unsrt}
\bibNote{Note: I used \textbf{Lyndon~White} as a pen-name up until February 2022. It thus appears as such in this section. I no longer use that name anywhere else.}
\bibliography{publications} 


\end{document}
